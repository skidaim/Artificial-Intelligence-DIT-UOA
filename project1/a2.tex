\documentclass{article}
\usepackage[LGR, T1]{fontenc}
\usepackage[greek]{babel}
\usepackage{cancel}
\usepackage{amsmath}
\begin{document}
\title{\textlatin{question2}}
\maketitle
\section*{ΕΚΦΩΝΗΣΗ:}

\subsection*{Πρόβλημα 2}
Έχετε το εξής πρόβλημα αναζήτησης:
\begin{itemize}
  \item Ο χώρος καταστάσεων αναπαρίσταται ως δένδρο.
  \item Ο κόμβος της ρίζας (αρχική κατάσταση) έχει τρεις κόμβους-παιδιά.
  \item Κάθε ένας από αυτούς τους κόμβους-παιδιά έχει επίσης τρεις κόμβους-παιδιάκ.ο.κ. Δηλαδή, το δένδρο έχει ομοιόμορφο παράγοντα διακλάδωσης ίσο με 3.
  \item Ο στόχος βρίσκεται στο βάθος 4.
\end{itemize}
Να υπολογίσετε θεωρητικά τον μικρότερο και το μεγαλύτερο αριθμό κόμβων που
επεκτείνονται από κάθε έναν από τους παρακάτω αλγόριθμους αναζήτησης,
υποθέτοντας ότι εκτελούν πλήρη αναζήτηση (δηλαδή, μέχρι να βρεθεί ο στόχος):
\begin{itemize}
  \item Αναζήτηση πρώτα κατά πλάτος (\(BFS\))
  \item Αναζήτηση πρώτα κατά βάθος (\(DFS\)). Υποθέστε ότι ο \(DFS\) εξετάζει πάντα πρώτα το αριστερότερο παιδί.
  \item Αναζήτηση με επαναληπτική εκβάθυνση (\(IDS\)).
\end{itemize}

\section*{ΑΠΑΝΤΗΣΗ:}
\subsection*{1. ΑΝΑΖΗΤΗΣΗ ΚΑΤΑ ΠΛΑΤΟΣ (\(BFS\))}
Ο αλγόριθμος \(BFS\) εξερευνεί το δέντρο επίπεδο ανά επίπεδο.\\
Δηλαδή, εξερευνεί κάθε κόμβο σε ένα επίπεδο πριν προχωρήσει στο επόμενο.\\
Άρα οι κόμβοι στα επίπεδα \(>\) 4 δεν μας ενδιαφέρουν.

\subsubsection*{ΘΕΩΡΗΤΙΚΑ ΜΕΓΑΛΥΤΕΡΟΣ ΑΡΙΘΜΟΣ ΚΟΜΒΩΝ}
Στην χειρότερη περίπτωση, ο \(BFS\) θα εξερευνήσει κάθε κόμβο στο επίπεδο 0, 1, 2, 3 και 4. Δηλαδή ο κόμβος στόχου είναι ο τελευταίος που εξερευνείται στο επίπεδο 4.\\
Επομένως, εφ'όσον βρισκόμαστε σε ένα τριαδικό δέντρο θα εξερευνήσει:

\[3^0 + 3^1+3^2+3^3+3^4 = 1 + 3 + 9 + 27 + 81 = 121 \text{ κόμβους,}\]

\subsubsection*{ΘΕΩΡΗΤΙΚΑ ΜΙΚΡΟΤΕΡΟΣ ΑΡΙΘΜΟΣ ΚΟΜΒΩΝ}
Ακόμα και στην καλύτερη περίπτωση, ο \(BFS\) θα εξερευνήσει κάθε κόμβο στο επίπεδο 0, 1, 2, 3. Όμως όταν φτάσει στο επίπεδο 4, ο κόμβος στόχου θα είναι ο πρώτος που θα συναντήσει. \\
Επομένως, εφ'όσον βρισκόμαστε σε ένα τριαδικό δέντρο θα εξερευνήσει:

\[3^0 + 3^1+3^2+3^3+1 = 1 + 3 + 9 + 27 + 1 = 41 \text{ κόμβους,}\]

\subsection*{2. ΑΝΑΖΗΤΗΣΗ ΚΑΤΑ ΒΑΘΟΣ (\(DFS\))}
Ο αλγόριθμος \(DFS\) απ'την άλλη, διαλέγει ένα \textlatin{branch} του δέντρου και το εξερευνεί όσο πιο βαθιά γίνεται, πριν κάνει backtrack και διαλέξει ένα άλλο branch. Επομένως, μας ενδιαφέρουν και οι κόμβοι σε επίπεδα \(>\) 4.\\
Ας ορίσουμε το βάθος του δέντρου του προβλήματος ως \(L\).

\subsubsection*{ΘΕΩΡΗΤΙΚΑ ΜΕΓΑΛΥΤΕΡΟΣ ΑΡΙΘΜΟΣ ΚΟΜΒΩΝ}
Στην χειρότερη περίπτωση, ο κόμβος στόχου θα είναι ο δεξιότερος κόμβος του επιπέδου 4 (καθώς ο \(DFS\) επιλέγει πάντα τον αριστερότερο κόμβο). \\
Σε αυτή την περίπτωση, ο \(DFS\) θα εξερευνούσε επιπλέον (εκτός απο κάθε κόμβο στα επίπεδα 0-4), \textbf{κάθε} υπόδεντρο του επιπέδου 4 (εκτός από το υπόδεντρο του κόμβου-στόχου, δηλαδή \(3^4 -1 = 80 \) υπόδεντρα) πριν τελικά βρει τον στόχο μέσω \textlatin{backtracking}. Κάθε υπόδεντρο, εφ'όσον είμαστε στο επίπεδο 4 θα έχει βάθος \(L-4\). Αυτό μας δίνει:

\[3^0 + 3^1+3^2+3^3+ 3^4 + (3^4-1)\cdot (3^1+\dots + 3^{L-5}) = \]
\[121 +80\cdot (3^1+\dots + 3^{L-4}) \]
κόμβους. (στο αθροισμα \((3^1+\dots + 3^{L-5})\) δεν περιλάβαμε το \(3^0\) διότι ήδη το μετράμε από τους κόμβους στον όρο \(3^4\)).

\subsubsection*{ΘΕΩΡΗΤΙΚΑ ΜΙΚΡΟΤΕΡΟΣ ΑΡΙΘΜΟΣ ΚΟΜΒΩΝ}
Στην καλύτερη περίπτωση, ο κόμβος στόχου θα είναι το αριστερότερο παιδί στο επίπεδο 4. Σε αυτή τη περίπτωση, ο \(DFS\), αφού εξερευνεί πάντα το αριστερότερο παιδί, θα εξερευνήσει:
\[1+1+1+1 + 1 = 5\]
κόμβους πριν βρει τον κόμβο στόχου (τον αριστερότερο κόμβο κάθε επιπέδου).

\subsection*{3. ΑΝΑΖΗΤΗΣΗ ΜΕ ΕΠΑΝΑΛΗΠΤΙΚΗ ΕκΒΑΘΥΝΣΗ (\(IDS\))}

O \(IDS\) εξερευνά τους κόμβους επίπεδο προς επίπεδο, παρόμοια με τον \(BFS\), αλλά επανεξετάζει τους κόμβους πολλές φορές καθώς αυξάνει το όριο βάθους. Εάν το δέντρο έχει \( L \) επίπεδα και ο στόχος βρίσκεται στον δεξιότερο κόμβο σε βάθος 4, ο \(IDS\) δεν θα επεκτείνει κόμβους κάτω από το βάθος 4, επειδή βαθαίνει προοδευτικά το όριό του και σταματά μόλις φτάσει στο στόχο.

\subsubsection*{ΘΕΩΡΗΤΙΚΑ ΜΕΓΑΛΥΤΕΡΟΣ ΑΡΙΘΜΟΣ ΚΟΜΒΩΝ}

Ο μεγίστος συνολικός αριθμός των κόμβων που εξερευνά ο \(IDS\) είναι το άθροισμα των κόμβων που εξερευνώνται σε κάθε όριο βάθους από το 0 έως το 4 (ο κόμβος στόχου είναι ο τελευταίος στο επίπεδο 4) : \\
\\
- Βάθος 0: 1 κόμβος. \\
- Βάθος 1: \( 1 + 3 = 4 \) κόμβοι. \\
- Βάθος 2: \( 1 + 3 + 9 = 13 \) κόμβοι. \\
- Βάθος 3: \( 1 + 3 + 9 + 27 = 40 \) κόμβοι. \\
- Βάθος 4: \( 1 + 3 + 9 + 27 + 81 = 121 \) κόμβοι. \\

Έτσι, ο συνολικός αριθμός των κόμβων που διερευνήθηκαν στη χειρότερη περίπτωση:

\[
1 + 4 + 13 + 40 + 121 = 179
\]

Επομένως, μέγιστοι κόμβοι που επεκτείνονται από το \(IDS\) = 179 κόμβοι.

\subsubsection*{ΘΕΩΡΗΤΙΚΑ ΜΙΚΡΟΤΕΡΟΣ ΑΡΙΘΜΟΣ ΚΟΜΒΩΝ}
Στην καλύτερη περίπτωση, ο στόχος είναι ο πρώτος κόμβος που θα συναντησει στο βάθος 4. Ακόμα και σε αυτή την περίπτωση, ο \(IDS\) θα πρέπει να περάσει από όλα τα προηγούμενα επίπεδα βάθους πριν βρει τον στόχο στο βάθος 4.

Όταν θα φτάσει στην 4η επεκταση βαθους και πηγαίνει έως το βάθος 4, θα βρει κατευθείαν τον κόμβο στόχου (θα εξερευνήσει 5 κόμβους, όπως στην καλύτερη περίπτωση \(DFS\)).   
     
- Σύνολο εξερευνηθέντων κόμβων:
\[
1 + 4 + 13 + 40 + 5= 63
\]

Επομένως, ελάχιστοι κόμβοι που επεκτείνονται από τον \(IDS\) = 63 κόμβοι.



\end{document}
