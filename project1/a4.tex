\documentclass{article}
\usepackage[LGR, T1]{fontenc}
\usepackage[greek]{babel}
\usepackage{cancel}
\usepackage{amsmath}
\begin{document}
\title{\textlatin{question4}}
\maketitle
\section*{ΕΚΦΩΝΗΣΗ:}

\subsection*{Πρόβλημα 4}
Θεωρήστε τον αλγόριθμο αμφίδρομης αναζήτησης που παρουσιάσαμε στις διαλέξεις.\\
Θεωρείστε ότι στα προβλήματα αναζήτησης που θα εφαρμοστεί ο αλγόριθμος
υπάρχει μοναδική κατάσταση στόχου.\\ Υποθέτουμε ότι τα ζευγάρια αλγορίθμων που
χρησιμοποιεί η αμφίδρομη αναζήτηση σαν υπορουτίνες για την (προς τα εμπρός)
αναζήτηση από την αρχική κατάσταση και την (προς τα πίσω) αναζήτηση από την
κατάσταση στόχου είναι:\\
(α) Αναζήτηση πρώτα σε πλάτος και αναζήτηση περιορισμένου βάθους\\
(β) Αναζήτηση με επαναληπτική εκβάθυνση και αναζήτηση περιορισμένου βάθους\\
(γ) Α* και αναζήτηση περιορισμένου βάθους\\
(δ) Α* και Α*\\
Είναι ο αλγόριθμος αμφίδρομης αναζήτησης με υπορουτίνες όπως στα (α)-(δ)
πλήρης; Είναι βέλτιστος; Ναι ή όχι και υπό ποιες συνθήκες.\\
Πως μπορεί να γίνει αποδοτικά ο έλεγχος ότι οι δύο αναζητήσεις συναντιούνται σε
κάθε μια από τις παραπάνω περιπτώσεις (α)-(δ);


\section*{ΑΠΑΝΤΗΣΗ:}
Αρχικά θα θεμελιώσουμε τα εξής:
\begin{itemize}
    \item Είναι προφανές ότι για να είναι το \textlatin{bidirectional search complete}, αρκεί ένας από τους αλγορίθμους να είναι \textlatin{complete}. Όσο και αν "ξεφύγει" ένας από τους 2 αλγορίθμους, αν είναι complete ο άλλος θα βρει την λύση αργά ή γρήγορα.
    
    \item Για να είναι \textlatin{optimal} το \textlatin{bidirectional search}, πρέπει (αλλα όχι αρκεί) \textbf{και οι 2 αλγόριθμοι} να είναι \textlatin{optimal}. \\

    \item Αν το \textlatin{branching factor b} δεν είναι πεπερασμένος, τότε κανένας αλγόριθμος δεν είναι πλήρης, άρα ποτέ ο \textlatin{Bidirectional} δεν είναι πλήρης. Επομένως, θα υποθέσουμε ότι το \textlatin{branching factor} είναι πάντα πεπερασμένο για να μην πάρουμε κάθε περίπτωση ξεχωριστά ενώ καταλήγουμε στο ίδιο αποτέλεσμα.
    
\end{itemize}
Με βάση αυτά + αυτά που γνωρίζουμε για τους παραπάνω αλγορίθμους (πότε είναι πλήρης, πότε είναι βέλτιστοι κλπ) θα απαντήσουμε.

\subsection*{(α) Αναζήτηση πρώτα σε πλάτος και αναζήτηση περιορισμένου βάθους}
\subsubsection*{Πλήρης?}
\textbf{Ναι}, αφού η αναζήτηση πρώτα σε πλάτος είναι πλήρης.



\subsubsection*{Βέλτιστος?}

\textbf{Όχι}, διότι η αναζήτηση περιορισμένου βάθους δεν είναι βέλτιστη, είτε το \textlatin{state space} είναι \latin{finite} είτε όχι. Αφού ο ένας από τους δύο δεν είναι βέλτιστος, τότε δεν είναι βέλτιστος ο \latin{bidirectional}.

\subsection*{(β) Αναζήτηση με επαναληπτική εκβάθυνση και αναζήτηση περιορισμένου βάθους\\}
\subsubsection*{Πλήρης?}
\textbf{Ναι}, αν το \textlatin{state space} είναι πεπερασμένο. Σε αυτή τη περίπτωση, η Αναζήτηση με επαναληπτική εκβάθυνση είναι πλήρης και άρα ο \textlatin{Bidirectional} είναι πλήρης.

Επιπλέον \textbf{ναι}, αν ο κόμβος στόχου βρίσκεται σε βάθος μικρότερο από το όριο βάθους της αναζήτησης περιορισμένου βάθους.

Άρα ο \textlatin{Bidirectional} δεν είναι πλήρης αν το \textlatin{state space} είναι άπειρο και ο κόμβος στόχου βρίσκεται σε βάθος μεγαλύτερο από το όριο βάθους της αναζήτησης περιορισμένου βάθους. 

\subsubsection*{Βέλτιστος?}
\textbf{Όχι}, διότι τόσο η αναζήτηση περιορισμένου βάθους όσο και η αναζήτηση με επαναληπτική εκβάυθνση  δεν είναι ποτέ βέλτιστες, είτε το \textlatin{state space} είναι \textbf{finite} είτε όχι καθώς και είτε ο κόμβος στόχου βρίσκεται σε \textlatin{reachable} βάθος είτε όχι.


\subsection*{(γ) Α* και αναζήτηση περιορισμένου βάθους}
\subsubsection*{Πλήρης?}
\textbf{Ναι}, αφού ο αλγόρθμος Α* είναι πλήρης.
\subsubsection*{Βέλτιστος?}

\textbf{Όχι}, διότι η αναζήτηση περιορισμένου βάθους δεν είναι βέλτιστη. Αφού ο ένας από τους δύο δεν είναι βέλτιστος, τότε δεν είναι βέλτιστος ο \textlatin{bidirectional}.

\subsection*{(δ) Α* και Α*}
\subsubsection*{Βέλτιστος?}
\textbf{Ναι}, αφού ο αλγόρθμος Α* είναι πλήρης.

\subsubsection*{Βέλτιστος?}
\textbf{Όχι}. Το γεγονός ότι θα συναντηθούν οι 2 Α* σε έναν κόμβο (ας πούμε Μ)  δεν σημαίνει ότι θα έχουν φτάσει στο Μ με τον βέλτιστο μονοπάτι, και ούτε καν ότι το πραγματικο βέλτιστο μονοπάτι από αρχή έως στόχο θα περιέχει το Μ. Η εξασφάλιση βέλτιστου \textlatin{bidirectional} Α* είναι πολύ πιο σύνθετο πρόβλημα.

\subsection*{Έλεγχος συνάντησης αλγορίθμων}
Για να διαπιστωθεί αποτελεσματικά αν οι δύο αναζητήσεις έχουν συγκλίνει, μπορεί να χρησιμοποιηθεί ένας πίνακας κατακερματισμού (\textlatin{hash table)} για να παρακολουθούνται όλοι οι κόμβοι που έχουν επισκεφθεί και οι δύο αλγόριθμοι. Κάθε φορά που ένας από τους δύο αλγορίθμους επισκέπτεται έναν κόμβο, ελέγχουμε αν ο κόμβος αυτός υπάρχει ήδη στο \textlatin{hash table}. Αν ναι, αυτό σημαίνει ότι οι δύο αναζητήσεις έχουν συναντηθεί- αν όχι, προσθέτουμε τον κόμβο στον πίνακα κατακερματισμού. Αυτή η προσέγγιση μας δίνει Ο(1) αναζήτηση για κάθε κόμβο, άρα είναι πολύ αποδοτικός.

\end{document}
